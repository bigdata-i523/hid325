\documentclass[sigconf]{acmart}

\usepackage{graphicx}
\usepackage{hyperref}
\usepackage{todonotes}

\usepackage{endfloat}
\renewcommand{\efloatseparator}{\mbox{}} % no new page between figures

\usepackage{booktabs} % For formal tables

\settopmatter{printacmref=false} % Removes citation information below abstract
\renewcommand\footnotetextcopyrightpermission[1]{} % removes footnote with conference information in first column
\pagestyle{plain} % removes running headers

\newcommand{\TODO}[1]{\todo[inline]{#1}}

\begin{document}
\title{The Importance of Data Sharing and the Replication of the Sciences, But What About Data Archiving?}




\author{J. Robert Langlois}
\affiliation{%
  \institution{Indiana University}
 \address{}
  \city{Bloomington, IN 47408} 
  \country{USA}}
\email{langloir@umail.iu.edu}




\begin{abstract}
With the rise of digital information, scientists have faced with many challenges when it comes to the topic of data management, including archiving and data sharing. while it is unproblematic to share and archive quantitative data, qualitative data remains a puzzle that social scientists need to solve when it comes to data sharing as well as data archiving. Many researchers are skeptical to engage in the practice of sharing digital retrieval due to privacy concern, fear of stigmatization, the problem of funding, repository of data, transparency, and so forth. While it is important to keep these challenges in mind, it is critical to take a look at the different advantages of data sharing and data archiving. 

\end{abstract}

\keywords{i523, HID325, Data Sharing and Data Archiving}


\maketitle



\section{Introduction}

The 21st century is witnessing a large influx of data due to the increase usage of technology. The digital data generated from scientific research is integral to the advancement of different scientific fields. While these data sets exist in large quantities, one challenge that many fields face is the lack of and/or prohibition of data being shared among researchers. Data sharing and its subsequent replication is a subject matter that is in dispute within in the sciences \cite{leetaru2016}. This is a significant area of contention in the United States; however, in other countries like the United Kingdom (U.K.), this issue has been addressed by making data sharing a matter of great importance; so much so, that the Joint Information Systems Committee of the U.K. made "data-sharing a priority, and has helped to establish a Digital Curation Centre…to be national focus for research and development into data issues" \cite{pryor2009skilling}. On one hand, opponents of data sharing are skeptical about it due to privacy concerns, fears of stigmatization, funding problems, repositories for data, transparency, etc.  Many researchers are open to sharing because it allows their work to be reviewed and creates opportunities to further their findings; however, the actual practice is stunted by researchers concerns \cite{nelson2009empty}. On the other hand, proponents of this practice continue to advocate for an open access policy that would allow data to quickly respond to societal problems and crises, as well as advance the sciences. While it is as important to keep in mind the different challenges to data sharing, like data archiving, sharing data among fellow scientists can be very beneficial, not only can this practice help to respond to crises more quickly, but also it can play a vital role in advancing science and research.

\section{figures}

In Figure \ref{f:fly} we show a fly. Please note that because we use
just columwidth that the size of the figure will change to the
columnwidth of the paper once we change the layout to final. CHnaging
the layout to final should not be done by you. All figures will be
listed at the end.

\begin{figure}[!ht]
  \centering\includegraphics[width=\columnwidth]{images/fly.pdf}
  \caption{Example caption}\label{f:fly}
\end{figure}

When copying the example, please do not check in the images from the
examples into your images directory as you will not need them for your
paper. Instead use images that you like to include. If you do not have
any images, do not dreate the images folder.

\section{Tables}

In case you need to create tables, you can do this with online tools
(if you do not mind sharing your data) such as
\url{https://www.tablesgenerator.com/} or other such tools (please
google for them). They even allow you to manage tables as CSV.

or generate them by hand while using the provided template in Table\ref{t:mytable}. Not ethat
the caption is before the tabular environment.

\begin{table}[htb]
\centering
\caption{My caption}
\label{t:mytabble}
\begin{tabular}{lll}
1 & 2 & 3 \\
\hline
4 & 5 & 6 \\
7 & 8 & 9
\end{tabular}
\end{table}

\section{The Relevance of Data Sharing and Data Replication }

\subsection{Data Sharing Helps to Respond to Crisis Quicker}

Sharing data among fellow scientists and researchers is crucial. By sharing data, researchers do not always have to start from scratch when they are responding to societal problems, such as medical epidemics, economic instability issues, natural disasters, etc.  As previously mentioned, an important aspect of data sharing is its ability to be used to respond to and help expeditiously resolve societal issues. \cite{yozwiak2015data}, advocated for data sharing among fellow researchers and scientists to quickly respond to outbreaks. The authors centered their arguments around the rise of Ebola back in April 2015 that raised serious panic around the whole world due to the dangerous impacts of the disease that could result in death from just one exposure. They explained how the rapid availability of research data had facilitated a more amenable response time to the rapidly spreading threat of Ebola. The data accessed allowed it to be determined that the virus had circulated from Guinea to Sierra Leone and that it was being sustained by person-to-person contact. The fact that the data was recoverable from the GenBank, a public database, allowed researchers ready access and assisted with tracking the source of the deadly virus; thus, leading to the advocacy for the sharing of data among researchers to allow for quicker responses to life-threatening crises. This is one example of how data sharing can have a crucial impact on the response time researchers have when responding to disasters that have a global impact. Moreover, \cite{vogel2014delays} wrote about the problem that public health policies faced when it came to responding to outbreaks like Ebola. He explained how bureaucracy and a lack of record keeping often delayed the ability of scientists to respond. A lack of collaboration among researchers can hinder progress when scientists need to respond to crises. Not only does data sharing help scientists to respond to and help provide critical solutions to outbreaks like Ebola, but access to the data can also contribute to the advancement of science through data replication. 

\subsection{Data Sharing and Replication Contribute to Increase the Availability of Digital Information}

Furthermore, data sharing and replication (the availability of multiple copies of a data set to different users) is an important practice that plays a crucial role in the advancing of the sciences. The way sciences grow is through scaffolding, which means that one must rely on the work of previously published research to come up with new findings; thus, further supporting the necessity for a collaborative effort among researchers and scientists. Data sharing has been shown to help spot errors in research. In 2013, for example, a graduate student pointed out a calculation error made by two Harvard professors. This discovery was only possible because the professors shared a spreadsheet of their research findings with the particular student \cite{leetaru2016}. In this case, and possibly in many other cases, data sharing has helped to build community among fellow researchers, uncover honest mistakes and, in worst-case scenarios, expose possible fraud. \cite{borgman2015if} made a series of observations about the relevance of sharing data and asked many poignant questions regarding content, parameters and the necessity for sharing information. One observation she made was that "science progressed for centuries without data sharing policies and then questioned, why is data sharing deemed so important to scientific progress now?" \cite{borgman2015if}. She challenged the notion of free and unhindered distribution of data and cautioned that preliminary questions must first be answered to determine what data to share, how much and in what context data should be released to advance the changes in sciences. Her point was that the data should not stand alone, but rather it ought to be accurately defined and contextualized within the means that identified, developed and synthesized the data into usable information. While Borgman's scrutiny has its place in the argument regarding the absence of policies—which she argues ought to be based on accurately defining the data parameters—and their ability to facilitate data sharing, it is also important to acknowledge that the various scientific fields have been faced with different types of data. 
Nowadays, scientists are being bombarded with an abundant presence of digital data, which has made it difficult to manage and store, and far more difficult to compare data exchanges to what it was centuries ago. If digital data that is generated through research is not being replicated, the world of science will face far more challenges in the years ahead due to a lack of evolving data to build upon. Data replication involves open access to data so that researchers can continually study, analyze, and make new discoveries about existing data. Now, if data sharing opens the door to the replication of scientific research and advancement, then why is there so much opposition to such a practice? Without replication, the sciences may become stagnant in their advancement of theories and potential solutions to global problems.

\subsection{Data Sharing Help Reduce Unnecessary Cost
}

Another reason to encourage data sharing is that data sharing help to reduce unnecessary cost. As digital information is made available in the cloud system, researchers will be able to access the same type of information at different locations.  ''Data sharing is driven by the need to maintain more accurate and up-to-date spatial databases, but at the same time reduce data acquisition and maintenance costs''\cite{stoakes2005data}. If data becomes more accessible, not only this will contribute to lowering the cost to access data, but also will it encourage researchers in their endeavors to respond to social outbreak quicker. Data sharing can also play a crucial role in advancing science, which is our next point. 

\subsection{Data Sharing Advances the Science}

Contrarily to what many might think, in order to advance science through replication of existing findings, scientists must rely on the work of previous researchers. Just like human being cannot live in vacuum; the way science develop is through collaborative effort among fellow scientists. One advantage of this practice is that by sharing their work, scientists will be able to spout errors that previous researchers have made, reveal fraud, build community, etc \cite{leetaru2016}. As found in \cite{borgman2012conundrum} policy makers must develop policies that explain how to embark in this process. If data are not being replicated, the world of science will face far more challenges in the future. Data replication involves open access to data so that researchers can continue to study, analyze, and make new conclusion about existing data. Now, if data sharing allows the replication of the sciences, why are many people opposed such practice? 

\section{Blocks to Data Sharing}

There seem to be many barriers to data sharing. One of the barriers to data sharing is transparency. 

\subsection{Transparency Issue}

Some researchers fear that their work is going to be poached and that they will not get credit for their findings, so they hold on to it and do not disclose it. The concerns of obscurity and/or credit being assigned to other researchers who might advance the original researcher's findings will cause a serious reluctance to sharing data. \cite{leetaru2016}, used the term "data parasites" to describe the practice of utilizing data without giving proper credit to prior publishers. Thus, to overcome this challenge, there should be honesty and allowance for intellectual probity, a sort of fair play between researchers where appropriate credit would be given. In addition to giving appropriate credit for findings, another important aspect of disclosure is the appropriate compensation for the release of intellectual property. Oftentimes, researcher have sacrificed their time, income potential, energies, and relationships to do the work they do; incentive needs to be provided to encourage the sharing of their findings that they have worked hard to develop.
Another concern pointed out by \cite{leetaru2016} is that data sharing may open the door for data analysts to disprove and/or scrutinize the work of the data producers. A potential solution that he proposed to this issue was co-authorship. He believed this would discourage the misuse of data by allowing collaboration among researchers. \cite{borgman2015if} also asserted that researchers ought to agree on the standards of practice needed to responsibly share data. She advocated that both data and its means of publication deserve equal status in scholarly communications to determine how to cite data in non-trivial ways. If data sharing and collaboration among researchers is to be effective, there need to be norms and regulations of how to do so. Collaboration among scientists can be a good thing to help mitigate and even eradicate the sense of fear that many researchers have in sharing their findings and the methodologies used to produce them. 

\subsection{The Problem of Patients' Privacy}

Another barrier to data sharing, specifically in the healthcare field, involves the protection of patient privacy; a lack thereof can lead to stigmatization and potentially hamper patients’ participation in healthcare research and treatment. \cite{yozwiak2015data}, highlighted some uncertainties that are involved data sharing, like whether data belongs to public or private domains. Still, another barrier is patient consent and their ability to fully understand how their participation can make them vulnerable to being potentially shunned and ostracized in their community based on their diagnosis and/or treatment. The researchers advocated for the responsible sharing of pertinent information among researchers to avoid this problem. It should also be mentioned that preclusion to the sharing of unnecessary information would also weaken the barriers to data sharing. Although data sharing is important, particularly during a medical outbreak, researchers ought to do their best to protect patient privacy to avoid any threat of stigmatization or isolation of patients. Rigorous ethical standards should be applied to safeguard patients’ privacy and dignity to allow for easier sharing of relevant data \cite{yozwiak2015data}. Shelton (2011) advanced that "Rather than viewing privacy concerns as impediment, policy makers, scientists and HIT specialists should embrace privacy as an opportunity that, if addressed, can enhance the flow of information." \cite{shelton2011electronic}. If patients' privacy is protected, this will ease and mitigate skepticism within those who are refusing to share their personal information for fear that their privacy will be violated. These steps in the research process can facilitate the progression of scientific research through the increase of public participation and collaboration. 
Besides addressing privacy concerns, researchers can focus on understanding what aspects of data need to be preserved and dispensed for the public good. As another potential barrier, data preservation and the awareness of what data needs to be preserved raises concerns about data quality, the absence of scientists to analyze data, and data storage options. Funding for research needs to be contingent upon the determination of the importance of digital data. Policies ought to be developed that relate to the use of data, such as what data to be preserved as well as what exceptions need to be made to data preservation. In addition, regulations about data hosts (warehouses for storing data) should be determined. For example, "agencies and the research community together need to create the digital equivalent of libraries: institutions that can take responsibility for preserving digital data and making them accessible over the long term" \cite{pryor2009skilling}.  Moreover, an effort to teach information management should be prioritized to facilitate data acquisition, data cleansing, data storage, and effective uses of data. While most scientific disciplines found that a data deluge is extremely challenging, great opportunities can be realized with better organization and open access to data \cite{economic22challenges}. It is important to train scientists, establish better policies to regulate data sharing, and increase the incentives for researchers from every fields to collaborate as they tackle the many issues that are faced by the modern sciences.
For data sharing and replication to be effective, scientists from diverse fields ought to come together because very rarely can progress happen in isolation. Currently, very few fields like astronomy, genomics, social sciences, and archaeology practice data sharing. The lack of success in implementing data sharing policies conveys the need for greater understanding of the roles of data in various sciences; highlighting the need to also seek the development of new models of scientific practice \cite{borgman2015if}. A new model can be in the arena of archiving; archiving data can be very expensive and difficult to manage, thus while it is encouraging that scientists share their work with each other, it is also crucial to have serious conversation about data housing, and the financial responsibility that involves in this practice. Until researchers come together to satisfy the response to those barriers, data sharing will remain a challenge among scientists. And if today's scientists and researchers do not make the effort to work together to facilitate effective and essential data sharing, future generations will experience the problem of lost data due to a lack of effective stewardship. 


\section{Saving Scientific Data for Future Generations
}

\subsection{Data Archival}

Along with the conversation surrounding data sharing and replication, another important conversation that needs to take place is around the infrastructure needed to archive data. While many people engage in a debate to encourage data sharing among scientists, the infrastructure to preserve the data does not yet fully exist. In reference to data, Nelson (2009) drew on a proverbial question, is it the chicken or the egg; what comes first, data sharing or the space to store data? He contends that while data sharing is encouraged among scientists, the infrastructure to store data is nonexistent and it is arduous task to pursue the development of it \cite{nelson2009empty}. Thus, it is tantamount to talk about data saving as we encourage data sharing because if scientists resort to sharing their findings then there also need to be a safe place to house the data. This preservation is critical for future generations to build upon the work that has already been done to advance scientific research. As the advocacy for data archiving increases, a new challenge arises: who will pay for all of this?
Serious conversation needs to continue to happen around data management. "Access to data requires that the data be hosted somewhere and managed by someone," \cite{berman2013will}. Although they acknowledged the effort of public and private sectors to archive data in certain fields like the life sciences, they also stipulated that many federally funded research data are at risk due to the lack of long-term structure that can ensure continual access and preservation of data. If data are not housed well, it could be said that a lot of efforts, money, and energies, are being wasted away due to lack of a secure and sustaining system of storage. \cite{lynch2008big} posited that scientists are not the best stewards of data and suggested the job of data archiving be entrusted to the institutions that employ the researchers. Ensuring that data is well preserved will lay the important groundwork for allowing data to be accessible in the future. Lynch also emphasized the idea that for data saving to be effective, collaboration between funders, institutions and scientists are crucial. He gave the example, such as the GenBank and the U.S.'s National Institutes of Health (NIH) genetic sequence database, as well as the U.S. National Virtual Observatory, to show the possibilities of what can be done.
\cite{berman2013will} proposed four approaches that can help improve the "partnership among sectors": 1) incentivize the private sector to be stewards of public research, 2) utilize the power of partnership between the public and private sectors to fund viable solutions, 3) create clear policies for the management of public data, and finally 4) encourage openness to new and diverse methods of research to advance public research abilities. Furthermore, they theorized that there should be adequate safeguards to prevent private sector’s control, access and use of public data. While these measures are applicable, there not be all great because by relinquishing the work of data saving to the private sector solely, it can create a very expensive problem to accessing data via private organizations that are highly incentivized by financial gains. It would probably be more effective if the federal government would pay for their own data scientists to be trained on how to effectively manage the data, and establish the requirements and expectations that all federally funded research remains in the public domain. 
\cite{sarkol2016scientific} corroborates the idea of licensing all research data to the public domain. This is the case, in the Netherlands, where for example, all the data retrievals are kept by the National Library. The U.S. can espouse this model by creating a center for data to be stored, and develop policies on how to access the data. This would prevent the private sector from having a monopoly on accessing, interpreting, sharing and store data that belongs in the public domain.  \cite{borgman2012conundrum} advanced that in its effort to encourage peer review, the National Science Foundation (NSF), makes data sharing a requirement in the grant contract, where researchers are required to submit a 2-page report of their research that can be used to facilitate peer review. This technique used by NSF was revealed to be a great example of how the federal government can overcome the conundrum of data sharing and archiving among scientists.


\section{Conclusion}

This document put forth the dialogue needed to assist researchers to address the different challenges they have experienced when it comes to data sharing and replication as well as data saving. The importance of data sharing has been explored, and examples how sharing information can help scientists to respond to global crises in a timely manner, like in the Ebola outbreak, have been provided. It has also been shown how data sharing and replication have helped to advance scientific research. It was postulated that in order for data to continue to exist, scientists need to embrace the idea of replicating their information. 
As research continues to occur and scientists increase their agreement to collaborating with one another, they will be better equipped to discover potential errors from previous retrievals, fix those errors and clean the data, and make other discoveries based on existing data. Scientists cannot operate as an Island. Policies need to be put in place, to understand what data to share, when to share, etc. For the continued advancement of the sciences, data sharing and archiving will require resources that facilitate the access, interpretation and maintenance of data. The importance of data sharing, data replication, and data archival cannot be overlooked. More discussion and new policies regarding how to share, store and replicate data is needed as well as effective parameters for how these processes will be funded and used in the future.



\begin{acks}

  The authors would like to thank Dr. Gregor von Laszewski for his
  support and suggestions to write this paper.

\end{acks}

\bibliographystyle{ACM-Reference-Format}
\bibliography{project} 

\appendix

We include an appendix with common issues that we see when students
submit papers. One particular important issue is not to use the
underscore in bibtex labels. Sharelatex allows this, but the
proceedings script we have does not allow this.

When you submit the paper you need to address each of the items in the
issues.tex file and verify that you have done them. Please do this
only at the end once you have finished writing the paper. To d this
cange TODO with DONE. However if you check something on with DONE, but
we find you actually have not executed it correcty, you will receive
point deductions. Thus it is important to do this correctly and not
just 5 minutes before the deadline. It is better to do a late
submission than doing the check in haste. 

\section{Issues}

\DONE{Example of done item: Once you fix an item, change TODO to DONE}

\subsection{Uncaught Bibliography Errors}

    \TODO{Missing bibliography file generated by JabRef}
    \TODO{Bibtex labels cannot have any spaces, \_ or \& in it}
    \DONE{Citations in text showing as [?]: this means either your report.bib is not up-to-date or there is a spelling error in the label of the item you want to cite, either in report.bib or in report.tex}

\subsection{Formatting}

    \DONE{Incorrect number of keywords and HID not included in the keywords}
    \DONE{Separate keywords with comma}

\subsection{Writing Errors}

    \DONE{Do not use the phrase {\em In this paper/report we show} instead use {\em We show}. It is not important if this is a paper or a report and does not need to be mentioned}

\subsection{Citation Issues and Plagiarism}

    \DONE{It is your responsibility to make sure no plagiarism occurs. The instructions and resources were given in the class}
    \DONE{Uncited Quotes}
    \DONE{Need to paraphrase long quotations (whole sentences or longer)}
    \DONE{Claims made without citations provided. e.g. in section 4}
    \DONE{The citation mark should not be in the beginning of the sentence or paragraph, but in the end, before the period mark. example: ... a library called Message Passing Interface(MPI) [7].}
    \DONE{Put a space between the citation mark and the previous word}

\subsection{Structural Issues}

    \DONE{Abstract is unnecesarily long}


\end{document}
